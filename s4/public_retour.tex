\documentclass[10pt]{scrartcl}
\usepackage[pretty]{mystd}
\title{}
\author{}
\date{}

\begin{document}
    %\maketitle
    \subsection*{Séries commutativement convergentes}
    Il s'agit des séries $\sum u_n$ telles que pour toute bijection $\varphi:\N\to\N$, la série $\sum u_{\varphi(n)}$ converge. 
    En particulier, une série commutativement convergente est convergente (en prenant $\varphi=\id_{\N}$).
    \begin{thm}
        Toute série $\sum u_n$ de réels absolument convergente est commutativement convergente. 
        De plus, pour toute bijection $\varphi:\N\to\N$, on a 
        \[
            \sum_{n=0}^\infty u_n=\sum_{n=0}^\infty u_{\varphi(n)}
        \]
    \end{thm}

    \begin{proof}
        Étant donné une bijection $\varphi:\N\to\N$, on remarque que 
        \[
            \forall N\geq 0,\ \sum_{n=0}^N|u_{\varphi(n)}|\leq \sum_{n=0}^\infty|u_n|<\infty\tag*{(1)}
        \]
        donc $\sum u_{\varphi(n)}$ converge absolument, donc converge. 
        
        Pour $\varepsilon > 0$, il existe $n_0\geq 0$ tel que $\sum_{n> n_0}|u_n|<\varepsilon/3$. 
        Il existe, en vertu de la surjectivité de $\varphi$, un entier $n_1$ tel que $\lbrace 0,\dots,n_0\rbrace\subseteq\lbrace\varphi(0),\dots,\varphi(n_1)\rbrace$ (on peut, par exemple, prendre pour $n_1$ le maximum des antécédants des entiers $0,\dots,n_0$ par $\varphi$).
        Donc 
        \begin{align*}
            \left|\sum_{n=0}^\infty u_n-\sum_{n=0}^\infty u_{\varphi(n)}\right| &\leq \left|\sum_{n=0}^\infty u_n-\sum_{n=0}^{n_0} u_n\right|+\left|\sum_{n=0}^{n_0} u_n-\sum_{n=0}^{n_1} u_{\varphi(n)}\right|+\left|\sum_{n=0}^{n_1} u_{\varphi(n)}-\sum_{n=0}^\infty u_{\varphi(n)}\right|\\
                                                                                &\leq 3\sum_{n>n_0}|u_n|<\varepsilon
        \end{align*}
    \end{proof}

    \begin{remarks}
        Dans la première question du deuxième exercice, on n'avait pas besoin de montrer le théorème pour conclure à la convergence de la série étudiée, pusqu'il suffisait d'utiliser la majoration (1).
    \end{remarks}

    La réciproque du théorème précédent est vraie. 
    C'est-à-dire que si une série réelle est commutativement convergente, alors elle est absolument convergente. 
    Cela vient du fait que si la série des valeurs absolues diverge, alors il est possible de construire une bijection qui contredit l'hypothèse de convergence commutative.

    Il est même possible de "faire pire", au sens du théorème suivant : 
    \begin{thm}[de Riemann]
        Soit une série $\sum u_n$ réelle qui converge, mais pas absolument.
        Alors 
        \[
            \forall x\in\R\cup\lbrace\pm\infty\rbrace,\ \exists\varphi\in\mathfrak S(\N),\ \sum_{n=0}^{+\infty}u_{\varphi(n)}=x
        \]
    \end{thm}

    \subsection*{Autre}
    \begin{exo}
        Soit $(u_n)$ une suite positive et décroissante. Prouver que si la série $\sum u_n$ converge, alors $nu_n\to 0$ à mesure que $n\to\infty$.
    \end{exo}

    \begin{proof}
        On procède par l'absurde. 
        En niant le résultat, on montre qu'il existe $\varepsilon>0$ tel que pour tout $N\geq 0$, il existe $q_N\geq N$ tel que $u_{q_N}\geq \varepsilon/q_N$.
        Pour des entiers $N$ et $q_N$ qui vérifient ce qui précède, on note $p_N=\lfloor q_N/2\rfloor\geq\lfloor N/2\rfloor$. 
        La décroissance de la suite $(u_n)$ donne
        \[
            \sum_{k=p_N}^{2p_N}u_k\geq  \sum_{k=p_N}^{2p_N}u_{2p_N}=(p_N+1)u_{2p_N}
        \]
        Mais $2p_N\leq q_N$ donc $u_{2p_N}\geq u_{q_N}\geq \varepsilon/q_N$, par décroissance de $u$ et choix de $q_N$.

        Mais aussi, $p_N+1=2p_N-p_N+1=2\lfloor q_N/2\rfloor -\lfloor q_N/2\rfloor+1> q_N-2-q_N/2+1$ donc $p_N+1\geq q_N/2$. 
        
        Finalement,
        \[
            \sum_{k=p_N}^{2p_N}u_k\geq \varepsilon/2
        \]

        Mais la suite $(p_N)_{N\geq 0}$ tend vers $+\infty$, il est ainsi possible de construire une extractrice $\varphi:\N\to\N$ telle que 

        \[
            \forall p\geq 0,\ \sum_{k=\varphi(p)}^{2\varphi(p)}u_k\geq \varepsilon/2\tag*{(1)}
        \]

        Mais $\sum_{k=\varphi(p)}^{2\varphi(p)}u_k=S_{2\varphi(p)}-S_{\varphi(p)-1}\to 0$ lorsque $p\to\infty$ ($S_n$ désigne la somme partielle d'ordre $n$).

        Un passage à la limite dans (1) montre alors $0\geq \varepsilon$, ce qui n'est pas.

    \end{proof}
\end{document}