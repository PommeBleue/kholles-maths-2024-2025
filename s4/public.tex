\documentclass[10pt]{scrartcl}
\usepackage[pretty]{mystd}
\title{}
\author{}
\date{}

\begin{document}
    %\maketitle
    \subsection*{Élève 1*}
    \begin{qc}
        Énoncer le CSSA.
    \end{qc}

    \begin{exo}
        Soit $(a_n)_{n\geq 1}$ une suite de réels positifs ou nuls tels que la série  $\sum a_n$ converge.
        \begin{enumerate}
            \item Montre que si $\alpha>1/2$, la série $\sum\frac{\sqrt{a_n}}{n^\alpha}$ converge.
            \item Que dire dans le cas $\alpha = 1/2$ ?
        \end{enumerate}
    \end{exo}

    \begin{exo}
        \hfill
        \begin{enumerate}
            \item Montrer que pour toute bijection $\varphi:\N^*\to\N^*$, la série $\sum 1/(n\varphi(n))$ converge, on notera $S(\varphi)$ sa somme. Montrer que $\varphi\mapsto S(\varphi)$ est majorée et déterminer son sup.
            \item Montrer que pour toute bijection $\varphi:\N^*\to\N^*$, la série $\sum \varphi(n)/n^2$ diverge, on notera $S_n(\varphi)$ sa somme partielle. Montrer que $\varphi\mapsto S_n(\varphi)$ est minorée et déterminer son inf.
        \end{enumerate}
    \end{exo}

    \subsection*{Élève 2}
    \begin{ccp}\hfill
        \begin{enumerate}
            \item Soient $(u_n)$ et $(v_n)$ deux suites réelles. On demandera à $(v_n)$ d'être non nulle a.p.d.c.r.
            \begin{enumerate}
                \item Prouver que si $u_n\sim v_n$ alors $u_n$ et $v_n$ sont de même signe a.p.d.c.r.
                \item Dans cette question, on suppose que $(v_n)$ est positive. 
                Montrer que si $u_n\sim v_n$ alors $\sum u_n$ et $\sum v_n$ sont de même nature.
            \end{enumerate}
            \item Nature de $\sum_{n\geq 2}\frac{((-1)^n+i)\sin(1/n)\ln n}{\sqrt{n+3}-1}$.
        \end{enumerate}
    \end{ccp}

    \begin{exo}
        Déterminer, pour $\alpha\in\R$, la nature de la série de terme général $u_n=\left(\frac1{4^n}\binom{2n}n\right)^\alpha$.
    \end{exo}

    \subsection*{Élève 3}
    \begin{ccp}\hfill
        \begin{enumerate}
            \item Démontrer la règle de d'Alembert pour le cas $<1$.
            \item Nature de $\sum_{n\geq 1}\frac{n!}{n^n}$.
        \end{enumerate}
    \end{ccp}

    \begin{exo}
        On considère $u_n=(n^3+6n^2-5n-2)/n!$.
        \begin{enumerate}
            \item Montrer que $\sum u_n$ converge. 
            \item Montrer que $\mathcal B=(1,X,X(X-1),X(X-1)(X-2))$ est une base de $\R_3[X]$ et décomposer $P=X^3+6X^2-5X-2$ dans cette base.
            \item En déduire $\sum_{n=0}^\infty u_n$.
        \end{enumerate}
    \end{exo}

    \subsection*{Élève 4}
    \begin{ccp}\hfill
        \begin{enumerate}
            \item Soient $(u_n)$ et $(v_n)$ deux suites réelles. On demandera à $(v_n)$ d'être non nulle a.p.d.c.r.
            \begin{enumerate}
                \item Prouver que si $u_n\sim v_n$ alors $u_n$ et $v_n$ sont de même signe a.p.d.c.r.
                \item Dans cette question, on suppose que $(v_n)$ est positive. 
                Montrer que si $u_n\sim v_n$ alors $\sum u_n$ et $\sum v_n$ sont de même nature.
            \end{enumerate}
            \item Nature de $\sum_{n\geq 2}\frac{((-1)^n+i)\sin(1/n)\ln n}{\sqrt{n+3}-1}$.
        \end{enumerate}
    \end{ccp}

    \begin{exo}
        Déterminer, pour $\alpha\in\R$, la nature de la série de terme général $u_n=\left(\frac1{4^n}\binom{2n}n\right)^\alpha$.
    \end{exo}

    \subsection*{Élève 5}
    \begin{ccp}
        On considère $u_n=\cos(\pi\sqrt{n^2+n+1})$.
        \begin{enumerate}
            \item Montrer qu'au voisinage de $+\infty$, on a $\pi\sqrt{n^2+n+1}=n\pi+\frac\pi2+\alpha\frac\pi n+\mathcal O(\frac1{n^2})$ où $\alpha$ est un réel que l'on déterminera.
            \item Montrer que la série $\sum_{n\geq 1}u_n$ converge. Converge-t-elle absolument ?
        \end{enumerate}
    \end{ccp}

    \begin{exo}
        Soit $(u_n)$ une suite positive et décroissante. Prouver que si la série $\sum u_n$ converge, alors $nu_n\to 0$ à mesure que $n\to\infty$.
    \end{exo}

    \subsection*{Élève 6}
    \begin{ccp}\hfill
        \begin{enumerate}
            \item Démontrer la règle de d'Alembert pour le cas $<1$.
            \item Nature de $\sum_{n\geq 1}\frac{n!}{n^n}$.
        \end{enumerate}
    \end{ccp}

    \begin{exo}
        Soit $\sum u_n$ une série à termes positifs convergente. 

        Montrer que la série $\sum_{n\geq 1}\frac{\sqrt{u_n}}n$ est convergente.
    \end{exo}
\end{document}