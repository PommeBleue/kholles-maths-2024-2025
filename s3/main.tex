\documentclass[10pt]{scrartcl}
\usepackage[pretty]{mystd}
\title{}
\author{}
\date{}

\begin{document}

    \begin{exo}[Compacts d'un espace de suites]
        On considère l'ensemble $E\subset \R^{\N}$ constitué des suites réelles bornées, qui est un $\R$-espace vectoriel.
        Pour une suite $u\in E$, on définit $\norm{u}=\sum_{n=0}^{+\infty}|u_n|2^{-n}$. Vérifier que $\norm{\cdot}$ définit une norme sur $E$ 
        et montrer que la partie $A=\lbrace u\in E,\ \forall n\in\N,\ u_n\in[0,1]\rbrace$ est compacte.
    \end{exo}

    \begin{proof}
        Cet exercice a été posé lors d'un oral du concours des Mines. 
        Il nécessite de savoir effectuer une extraction diagonale lorsqu'on dispose d'une famille dénombrable de suites (en classe, l'extraction diagonale sur un nombre fini de suites a été vue).

        On résout l'exercice en plusieurs étapes. 
        Il nous est demandé de montrer que dans $(E,\norm{\cdot})$, l'ensemble $A$ est compact, c'est-à-dire, que toute suite d'éléments de $A$ admet une suite extraite convergente de limite un élément de $A$. 

        Soit alors une suite $(u^p)_{p\in\N}\in A^{\N}$. 
        Le but est de trouver une suite $u$ telle qu'il existe une extractrice $\varphi$ telle que $\norm{u^{\varphi(p)}-u}\to 0$ quand $p\to\infty$ (autrement dit, une valeur d'adhérence).
        \subsubsection*{Étape 1 : Intuition du candidat pour être une valeur d'adhérence} Voyons voir ce que $\norm{u^{\varphi(p)}-u}\to 0$ nous donne, si on suppose qu'on dispose d'une valeur d'adhérence $u$ et une extractrice $\varphi$ telle que l'extraction de $(u^p)$ par $\varphi$ tend vers $u$. 
        On peut faire une observation simple : 
        \[
            \forall n\in\N,\forall p\in\N,\ |u^{\varphi(p)}_n-u_n|2^{-n}\leq\norm{u^{\varphi(p)}-u}
        \]
        Donc pour $n\in\N$, $u_n$ est une valeur d'adhérence de la suite $(u_n^p)_{p\in\N}$. C'est comme ça que l'on va construire notre candidat. 

        \subsubsection*{Étape 2 : Extraction diagonale} 
        Par hypothèse, pour tout $p\in\N$, $u_0^p$ est un élement de $[0,1]$, la suite $(u_0^p)_{p\in\N}$ est donc une suite bornée, elle admet donc, par Bolzano-Weierstrass, une suite extraite $(u_0^{\varphi_0(p)})_{p\in\N}$ convergente de limite $u_0\in[0,1]$.
        De la même façon, la suite $(u_1^{\varphi_0(p)})_{p\in\N}$ est encore à valeurs dans $[0,1]$, et admet donc, toujours par Bolzano-Weierstrass, une suite extraite $(u_1^{\varphi_0\circ\varphi_1(p)})_{p\in\N}$ convergente de limite $u_1\in[0,1]$. 
        On itère ce procédé pour créer une suite $(\varphi_n)_{n\in\N}$ d'extractrices et $u\in[0,1]^{\N}$ vérifiant 
        \[
            \forall n\in\N,\ (u_n^{\varphi_0\circ\cdots\circ\varphi_n(p)})_{p\in\N}\text{ converge et a pour limite }u_n
        \]
        On veut à présent définir une extractrice $\varphi$ telle que 
        \[
            \forall n\in\N,\ (u_n^{\varphi(p)})_{p\in\N}\text{ converge et a pour limite }u_n
        \]
        On vérifie que $\varphi$ définie par $\varphi(n)=\varphi_0\circ\cdots\circ\varphi_n(n)$ suffit.

        \subsubsection*{Étape 3 : On montre que notre suite extraite ainsi construite converge bien vers notre candidat}
        Autrement dit, on montre que $\norm{u^{\varphi(p)}-u}\to 0$ quand $p\to\infty$. Soit $\varepsilon >0$. Pour $p\in\N$, on a 
        \[
            \norm{u^{\varphi(p)}-u}=\sum_{n=0}^\infty\frac{|u^{\varphi(p)}_n-u_n|}{2^n}
        \]
        On montre que cette somme est plus petite que $\varepsilon$ pour $p$ assez grand. 
        Déjà, $2^{-n}\leq \varepsilon/4$ pour $n$ assez grand, notons donc $n_0$ un rang à partir duquel cela est vrai. On a 
        \[
            \sum_{n=0}^\infty\frac{|u^{\varphi(p)}_n-u_n|}{2^n}=\sum_{n=0}^{n_0}\frac{|u^{\varphi(p)}_n-u_n|}{2^n}+\sum_{n>n_0}\frac{|u^{\varphi(p)}_n-u_n|}{2^n}
        \]
        Pour tout $n\in\N$, $|u^{\varphi(p)}_n-u_n|\leq 2$, d'où 
        \[
            \sum_{n>n_0}\frac{|u^{\varphi(p)}_n-u_n|}{2^n}\leq 2^{-n_0+1}\sum_{n\geq 1}2^{-n}\leq\varepsilon/2
        \]
        Il reste à montrer que la première somme est plus petite que $\varepsilon/2$ pour $p$ assez grand. 
        Notons, pour tout $n\in\N$, $p_n(\varepsilon/4)$ un entier vérifiant
        \[
            \forall p\geq p_n(\varepsilon/4),\ |u^{\varphi_0\circ\cdots\circ\varphi_n(p)}_n-u_n|\leq\varepsilon/4
        \]

        On pose $q=\max(p_0(\varepsilon/4),\dots,p_{n_0}(\varepsilon/4))$ et soit $p\geq q$. Alors, pour $n\leq n_0$, $\varphi(p)=\varphi_0\circ\cdots\circ\varphi_n(\varphi_{n+1}\circ\cdots\circ\varphi_{n_0}(p))$, mais comme $\varphi_{n+1}\circ\cdots\circ\varphi_{n_0}$ est strictement croissante comme composée de telles applications, $\varphi_{n+1}\circ\cdots\circ\varphi_{n_0}(p)\geq p\geq q$, donc 
        \[
            |u^{\varphi(p)}_n-u_n|\leq\varepsilon/4
        \]
        De sorte que
        \[
            \sum_{n=0}^{n_0}\frac{|u^{\varphi(p)}_n-u_n|}{2^n}\leq \frac{\varepsilon}4\underbrace{\sum_{n=0}^{n_0}2^{-n}}_{\leq 2}\leq\varepsilon/2
        \]
        Finalement, pour $p\geq q$, 
        \[
            \norm{u^{\varphi(p)}-u}=\sum_{n=0}^{n_0}\frac{|u^{\varphi(p)}_n-u_n|}{2^n}+\sum_{n>n_0}\frac{|u^{\varphi(p)}_n-u_n|}{2^n}\leq\varepsilon/2+\varepsilon/2=\varepsilon
        \]
    \end{proof}

    \begin{remarks}
        \begin{enumerate}
            \item Comme on le voit dans la première "étape" de la démonstration (qui n'est pas vraiment une étape et qu'on ne ferait certainement pas figurer dans une rédaction en temps normal), même s'il est normal de ne pas avoir tout de suite les grandes idées de la démonstration, notamment dans le contexte d'une khôlle, on peut à tout le moins réfléchir à la forme d'un candidat valeur d'adhérence en écrivant ce que ça veut dire d'être valeur d'adhérence. 
            On voit qu'en faisant ça, on comprend qu'on devra construire un candidat $u\in A$ tel que pour tout $n\in\N$, $u_n$ est valeur d'adhérence de la suite numérique $(u^{p}_n)_p$, et on devra même avoir mieux : il faudra construire une extractrice $\varphi$ telle que pour tout $n\in\N$, on ait 
            \[
                u_n^{\varphi(p)}\to u_n
            \]
            et c'est là que l'idée d'une extraction diagonale devrait venir, sachant la forme des suites composant $A$.
            \item Une question bonus qu'on pourrait poser une fois la démonstration du résultat de l'exercice faite est : est-ce que la norme $\norm{\cdot}$ est équivalente à la norme infinie sur $E$ ? Je vous laisse y réfléchir (utiliser le résultat de l'exercice pour répondre à cette question).
        \end{enumerate}
    \end{remarks}

    \newpage 
    \begin{exo}[Adhérence d'un graphe]
        Soit $f:\Rpe\to\R$ l'application définie par $f(x)=\cos(1/x)$. 
        On pose $A=\lbrace (x,f(x)),\ x>0\rbrace\subset\R^2$. Déterminer l'adhérence de $A$.
    \end{exo}

    \begin{proof}
        Une partie peut-être "difficile" de l'exercice est de réussir à deviner l'adhérence de $A$, mais une fois le dessin fait, on voit bien qu'on sera amené à montrer que $\overline A= \lbrace 0\rbrace\times[-1,1]\coprod A$.
        En effet, on sait, vu la caractérisation séquentielle de l'adhérence, qu'on peut décrire l'adhérence d'un ensemble grâce à des limites de suites à éléments dans cet ensemble. 
        En imaginant des suites de points sur le graphe, on voit que les limites possibles sont soit les points du graphe, soit la bande $\lbrace 0\rbrace\times [0,1]$ où il y a une accumulation de points du graphe.
        \begin{center}
            \begin{tikzpicture}
                \draw[->] (0,0) -- (1.5,0) node[right] {$x$};
                \draw[->] (0,-1) -- (0,1.2) node[left] {$y$};
                \draw[thick,color=red] (0,-1) -- (0,1) node [rotate=90,midway,above] {\tiny Accumulation};
                \draw[color=BoxColor1,domain=0.05:1.5, samples=5000, variable=\x] plot ({\x}, {cos ( 1 / \x r)});
            \end{tikzpicture}
        \end{center}
        On va montrer que $\overline A= \lbrace 0\rbrace\times[-1,1]\coprod A$ (dans $\R^2$ muni de la norme infinie).
        \begin{itemize}
            \item[$\boxed{\subset}$] Pour ce sens, il suffit de montrer que l'ensemble de droite (appelons-le $B$) est un fermé.
                Comme il contient trivialement $A$, la conclusion sera directe. Montrons alors que $B^c$ est ouvert, et soit alors $(x,y)\in B^c$.
                On discrimine selon que 
                \begin{enumerate}[1.]
                    \item $|y|>1$, et comme tout élément $(\alpha,\beta)\in B$ vérifie $|\beta|\leq 1$, il est clair que $B((x,y),|y|-1)\subset {}^cB$.
                    \item $|y|\leq 1$. Dans ce cas, nécessairement $x\neq 0$, et alors $x>0$ ou $x<0$.
                    Le cas $x<0$ se traite de la même manière qu'en 1. Supposons alors $x>0$. Notons $[a,b]$ ($a<b$) un segment de $\Rpe$ centré en $x$.
                    $f_{|[a,b]}$ est bien définie, encore continue, et alors le graphe de cette fonction est un fermé.
                    En effet, si $(x_n)\in[a,b]^{\N}$ est telle que $((x_n,f(x_n)))_n$ converge, alors $(x_n)$ et $(f(x_n))$ convergent, de limites respectives $\alpha$ et $f(\alpha)$ car 
                    $f$ est continue ; $[a,b]$ étant fermé, $\alpha\in[a,b]$ et donc $(\alpha,f(\alpha))$, limite de $((x_n,f(x_n)))_n$, est un point du graphe de $f_{|[a,b]}$.
                    Notons $G$ le graphe de $f_{|[a,b]}$, et $r=\min(d((x,y),G),b-a)$. $G$ étant fermé et ayant $(x,y)\notin G$, on a $r>0$.
                    Alors $B\left((x,r),\frac r2\right)\subset B^c$. En effet, si $(u,v)\in B\left((x,y),\frac r2\right)$, pour tout $t\in[a,b]$, 
                    \[
                        \norm{(u,v)-(t,f(t))}\geq \norm{(t,f(t))-(x,y)}-\norm{(u,v)-(x,y)}\geq r-\frac r2=\frac r2>0
                    \]
                    et comme, par choix de $r$, $u\in[a,b]$, si $t\notin[a,b]$, alors $(u,v)\neq(t,f(t))$, mais aussi $u\neq 0$ ; finalement $(u,v)\notin B$.
                \end{enumerate}
                \item[$\boxed\supset$] Il suffit de construire, étant donné $\alpha\in[-1,1]$, une suite $(x_n)\in\Rpe$, telle que $x_n\to 0$ et $(x_n,f(x_n))\to (0,\alpha)$ (i.e $f(x_n)\to \alpha$).
                $\alpha\in\cos(]2n\pi,2(n+1)\pi])$ pour tout $n\in\N$, notons alors $y_n\in]2n\pi,2(n+1)\pi]$ tel que $\cos(y_n)=\alpha$. 
                Clairement $y_n>0$ pour tout $n\geq 0$ et $y_n\to +\infty$. On pose $x_n=\frac1{y_n}>0$ pour tout $n\in\N$. 
                $x_n\to 0$ et $f(x_n)=\cos(y_n)=\alpha\to\alpha$.
        \end{itemize}
    \end{proof}

    \begin{remarks}
        On voit que la démonstration précédente est faite en utilisant comme norme la norme infinie sur $\R^2$ (c.à.d $\norm{(x,y)}=\max(|x|,|y|)$), ce qui n'a pas été précisé dans l'énoncé tel que je vous l'ai donné. 
        Ce manque de précision est une erreur de ma part tant le programme de la semaine n'incluait pas le résultat d'équivlence des normes en dimension finie dont la démonstration utilise des notions de continuité. 
        Cela dit, si on ne sait pas ce résultat, le premier réflèxe à avoir est de se demander "pour quelle norme ?", et si on sait le résultat, alors il n'y a aucun problème pour comprendre l'énoncé et résoudre l'exercice.
    \end{remarks}

\end{document}