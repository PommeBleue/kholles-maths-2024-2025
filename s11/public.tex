\documentclass[10pt]{scrartcl}
\usepackage[pretty]{mystd}
\title{}
\author{}
\date{}

\begin{document}
    \subsection*{Élève 1*}
    \begin{exo}
        Soit $G$ un sous-groupe de $\gl_n(\C)$ borné. 
        Montrer que les valeurs propres des matrices de $G$ 
        sont toutes de module $1$. 
        Si de plus $G$ est supposé inclus dans $B(I_n,\sqrt2)$
        pour une certaine norme d'opérateur, montrer que $G$
        est trivial.
    %     Soit $A\in\mathcal M_n(\C)$.
    %    \begin{enumerate}
    %     \item Si $A$ est inversible, montrer que $A$ 
    %     est diagonalisable si et seulement si $A^2$ l'est.
    %     \item Montrer que $A$ est diagonalisable si 
    %     et seulement si $A^2$ l'est et que $\ker A = \ker A^2$.
    %     Que dire si $A\in\mathcal M_n(\R)$ ?
    %    \end{enumerate}
    \end{exo}

    \subsection*{Élève 2}
    \begin{ccp}
        On pose $A=\left(\begin{array}{c c} 2 & 1\\ 4 & -1\end{array}\right)$.
        \begin{enumerate}
            \item Déterminer les valeurs propres et vecteurs propres associés de $A$.
            \item Déterminer toutes les matrices qui commutent avec $A$.
        \end{enumerate}
    \end{ccp}

    \begin{exo}
        L'équation 
        \[
            X^2=\left(\begin{array}{c c c} 0 & 1 & 0\\ 0 & 0 & 1\\ 0 & 0 & 0\end{array}\right)
        \]
        possède-t-elle des solutions dans $\mathcal M_3(\C)$ ? 
    \end{exo}

    \subsection*{Élève 3}
    \begin{ccp}
        On considère 
        \[
            A=\left(\begin{array}{c c c}
                0 & 2 & -1\\
                -1 & 3 & -1\\
                -1 & 2 & 0
            \end{array}\right)
        \]
        \begin{enumerate}
            \item Déterminer les valeurs propres de $A$. 
            \item $A$ est-elle inversible ? diagonalisable ?
            \item Déterminer $\mu_A$.
            \item Calculer les puissances de $A$.
        \end{enumerate}
    \end{ccp}

    \begin{exo}
        Quelle est la trace d'une matrice nilpotente ? 
        Donner un contre-exemple à la réciproque. 
        Si $A$ est une matrice de $\mathcal M_2(\C)$ 
        telle que $\tr(A)=\tr(A^2)=0$, montrer que $A$ 
        est nilpotente.
    \end{exo}
\end{document}